\documentclass[11pt]{article}
% packages from texlive 2012
\usepackage{booktabs}
\voffset=-0.75in \hoffset=-0.75in
\addtolength{\textheight}{1.5in}
\addtolength{\textwidth}{1.5in}
\newcommand{\clinchor}{{\tt clinchor}}
\newcommand{\SDDS}{{\tt SDDS}}
\newcommand{\sddshist}{{\tt sddshist}}
\newcommand{\elegant}{{\tt elegant}}
\begin{document}
\title{User's Guide to Program \clinchor}
\author{L. Emery}
\maketitle
%\tableofcontents
\section{Introduction}
The program \clinchor\ calculates the growth rates of longitudinal and
transverse coupled bunch modes in an electron storage ring.  This
program combines features that are found separately in other programs
such as ZAP\cite{ZAP}, BBI\cite{BBI} and PC-BBI\cite{PC-BBI}.  The
effort in writing another coupled bunch program is justified further
by the opportunity to implement more flexible input and output
methods. The input command file is written entirely in namelist
commands, and the input and output data files are in
SDDS format.  The postprocessing and graphics are
done outside of \clinchor\ using the \SDDS\ toolkits.

The method of calculation in \clinchor\ is based on the normal mode
analysis of K. Thompson and R. Ruth\cite{Thompson:PAC89}. 
The bunches are treated as point charges and therefore only rigid bunch
modes are calculated, which means Landau damping and coherent damping from
short-term wake-fields are not considered.
The effect of bunch length on the strength of the wake fields generated is
taken into account, however.

For those with inquiring minds, the name of the program comes from the expression
``{\bf c}alculation of {\bf l}ongitudinal and transverse coupled bunch {\bf in}stability
due to {\bf c}avity {\bf h}igher-{\bf or}der mode resonators.''

The rest of the guide is organized in three sections. The general
features are presented in the next section.  In section
\ref{sect:namelists}, the input commands are explained in more detail.
The HOM definition file format is explained in section \ref{sect:HOM
file definition}.

\section{General Features}
The calculation features of the program are:
\begin{enumerate}
    \item treatment of general bunch distributions plus randomization of charge,
    \item inclusion of bunch form factor, which reduces the effect of
      impedance on the beam,
    \item shifting of one or more HOM frequencies to the closest dangerous resonant frequency,
    \item sweeping of one cavity higher-order mode (HOM) frequencies,
    \item randomization of the cavity higher-order mode (HOM) frequencies for Monte Carlo simulations, i.e representing changes that might occur during operations,
    \item frequency-stepping (staggering) of the HOM frequencies across cavities,
    \item effect of the beta function at the RF cavities for transverse motion.
\end{enumerate}
The implementation features of the program are:
\begin{enumerate}
    \item {\tt C} language coding,
    \item input file with namelist commands,
    \item HOM input data stored in convenient \SDDS-format files, along with flags for number of cavities and setting to resonance,
    \item output files compatible with \SDDS\ toolkits,
\end{enumerate}

These items are briefly explained below.

\subsection{General Bunch Distribution}

Programs like ZAP and BBI, and the theory upon which they are based
can only treat symmetric bunch distribution. Thompson and
Ruth\cite{Thompson:PAC89} developed a simple theory to calculate
modes of irregularly spaced bunches. Their formalism also allows the
possibility of bunches of different charge.  In order to get the
flexibility of having the most general bunch patterns, program
\clinchor\ has three commands to generate a symmetric bunch pattern, a
bunch train, and a general bunch pattern. See namelist commands {\tt
symmetricBunchPattern}, {\tt bunchTrain}, and {\tt
generalBunchPattern} in section \ref{sect:namelists}. The bunch
pattern can be read from a file using {\tt
bunchPatternFromFile}. Namelist command {\tt randomizeBunchCurrent}
generates a random component to the bunch distributions after the beam
is defined.

\subsection{Bunch Form Factor} 

Bunches with non-zero length have power spectral densities that decay with
frequency. 
This will decrease the effect of the impedances, which improves stability
expectations. 
The program can read a file of rms bunch lengths as a function of
current for determining the form factors for each bunch in the growth rate
calculations. 
It is assumed that the longitudinal distribution of the bunches is
Gaussian.

\subsection{Shifting of the HOM Frequencies to the Closest Dangerous Resonant Frequency}
A monopole HOM impedance can contribute the maximum growth rate to a
symmetric longitudinal mode when the HOM frequency falls on a
revolution harmonic upper-synchrotron sideband.  Similarly, a dipole
HOM impedance can contribute the maximum growth rate to a symmetric
transverse mode when the HOM frequency falls on a revolution
harmonic lower-betatron sideband.  An optional data column may be
included in the HOM definition file (see section \ref{sect:HOM file
definition}) as a flag to force selected HOM frequencies onto dangerous
resonances.  Because of the relatively low probability for HOM
frequencies to hold exactly these resonant values, this option is
mostly useful in checking the calculated growth rates against those
obtained in hand calculations.

\subsection{Sweeping of the HOM Frequencies}
The setup command {\tt sweepFrequency} allows one to sweep the
frequency of a selected HOM through some range.  With a little
postprocessing, a plot of the growth rate as a function of that HOM
frequency may be obtained.  This is a feature borrowed from BBI.
The results can be used to cross-check with formulas of single HOM.

\subsection{Randomization of the HOM Frequencies}
This optional feature (borrowed from PC-BBI) allows one to make a
Monte Carlo simulation of the CBM growth rates using the HOM
frequencies as random numbers. One can then determine the probability
of instability for the set of HOMs defined. See namelist command {\tt
randomizeHOMFrequencies}.

\subsection{Staggering of the HOM Frequencies}

The frequency stepping (staggering) of the HOM frequencies in an RF system
reduces the probability that two or more HOM impedances will contribute to
the same coupled bunch mode.  
The staggering is accomplished by varying
the cavity dimensions in uniform steps from cavity to cavity. The
staggering is expected to vary with HOM type.  Staggering steps per
cavity are entered as an optional column in the HOM definition file.

\subsection{Beta Functions at the RF Cavities}
This concerns only the transverse coupled bunch mode calculations.
All previous calculations of coupled bunch motion that I know (ZAP,
BBI) assume a constant beta function around the ring (smooth focusing
approximation).  
There is no reason not to use the correct beta function values (it is not
that complicated), therefore \clinchor\ uses and requires values of the
beta functions at the RF cavities.

\subsection{{\tt C} Language}
The usual advantages of a {\tt C} code versus Fortran code applies here (as of the writing in 1992; Fortran may catch up in future years).
In  {\tt C}, one can dynamically allocate memory for structure arrays and
matrices as the calculation requires them. The memory is freed when it
is no longer needed. This results in a smaller executable file. {\tt C}
code, in general, is easily ported to other types of computers. 
(So far the program has been ported to various Unix, Linux, and Windows OSs
since 1992.) 
The code is written in a verbose style (i.e., variable names are based on
un-abbreviated physical quantity names), which makes it easy for anyone to
understand the code.

\subsection{Command Input File}
To run \clinchor\ one needs are least two input files. One containing
the namelist commands, and one for the HOMs definitions.  The input
file containing the namelist commands is specified on the shell
command line as in this example,
\begin{verbatim}
clinchor input.cb
\end{verbatim}
where {\tt input.cb} is the input file. The complete list of namelist
commands and variables is given in section \ref{sect:namelists}.

Macros (tag-value pairs) can be defined on the command line for substituting strings that appear in the input file between ``$<$'' and ``$>$'' characters. For example,
\begin{verbatim}
clinchor input.cb -macro=totalCurrent=102,dipoleHOMs=dipole.sdds
\end{verbatim}
where tags $<${\tt totalCurrent}$>$ and  $<${\tt dipoleHOMs}$>$ might appear in the input file.

\subsection{HOM Definition File}
The HOM data is contained in a \SDDS\ format file rather than
appearing explicitly in the input command file.  The choice of the
\SDDS\ format facilitates the organization of the HOM data.


\subsection{Output files}
The main output file is directed to the standard
output by default. The file echoes the namelist commands and contains some
secondary physical quantities, which the user should verify to ensure
that no data entry error occurred.  

Other output files are generated for specific calculations.


%%%%%%%%%%%%%%%%%%%%%%%%%%%%%%%%%%%%%%%%%%%%%%%%%%%%%%%%%%%%%%%%%%%%%%%%%%%%%%%%
\section{List of Namelist Commands} \label{sect:namelists}
%%%%%%%%%%%%%%%%%%%%%%%%%%%%%%%%%%%%%%%%%%%%%%%%%%%%%%%%%%%%%%%%%%%%%%%%%%%%%%%%
The namelist commands in the command input file define the whole
problem: the ring parameters, the bunch distribution of the beam, the
cavity HOM impedances, the optional instructions to setup the
calculation, and the calculation itself. Each namelist commands use
variables which further specify the command.

The namelist commands are processed by subroutines from object
libraries made available publicly by M. Borland.

Below, the namelist commands in \clinchor\ are listed and grouped according to their function:
\begin{center}
\begin{tabular}{ll} \toprule
Basic ring parameter            & \tt ringParameters               \\ \hline
Cavity HOM definitions          & \tt monopoleHOMs                 \\ 
                                & \tt dipoleHOMs                   \\ \hline
Beam definition                 & \tt symmetricBunchPattern       \\ 
                                & \tt bunchTrain                   \\
                                & \tt generalBunchPattern         \\ 
                                & \tt bunchPatternFromFile         \\
                                & \tt randomizeBunchCurrent         \\ \hline
Option setup commands           & \tt randomizeHOMFrequencies         \\ 
                                & \tt sweepFrequency               \\ \hline
Calculation commands             & \tt doLongitudinalMotion        \\ 
                                & \tt doTransverseMotion          \\ \bottomrule
\end{tabular}
\end{center}

The following pages describe all the namelist commands. For each
command, one will find:
\begin{itemize}
\item the type and function of the command,
\item a command definition listing, and
\item a detailed explanation of the variables of the command.
\end{itemize}

The command definition listing is of the form
\begin{verbatim}
&<command-name>
    <variable-type> <variable-name> = <default-value>
    .
    .
    .
&end
\end{verbatim}
The component {\tt <variable-type>} can be one of three types:
\begin{itemize}
\item {\tt double} --- for a double-precision type, used for most
physical quantity variables,
\item {\tt long} --- for an integer type, used for some physical
    quantity variables, such as the harmonic number, but most often
    for the logical flag variables, where 0 mean false and 1 means
    true.
\item {\tt STRING} --- for a character string enclosed in double
quotes, used most often for a filename variable.
\end{itemize}
Note that the component {\tt <variable-type>} appears in these pages
only as a place-holder, and shouldn't be used literally in an actual input
file.\footnote{macro place-holders in the input file would have explicit angle brackets enclosing a tag, as explained above.}  The namelist in an input file should look like this:
\begin{verbatim}
&<command-name>
    [<variable-name> = <default-value>]
    [<array-name[<index>] = <value>,[,<value> ...],]
&end
\end{verbatim}
The square brackets denotes an optional component. Not all variables
need to be defined -- the defaults may be sufficient.  Those that do
need to be defined are noted in the detailed explanations.  The only
variables that don't have default values in general are string
variables.

Array variables take a list of values with the first one assigned to
the array element {\tt index}.  If the index value isn't given, then the
namelist processor assumes the first array element is number zero.
The case of the letters in all namelist and variable names is
important.

One will note that quantities which aren't traditionally described in
MKS units have their units appearing explicitly in their names.  This
convention is used to avoid confusion about which non-MKS units to
use.

Whenever a namelist is read, it is written back to the standard output
file, which is the screen device unless the output is redirected.

In general, the namelist commands can be repeated as many times as one
wishes. The reason for repeating may be to change a variable value
between calculation commands, to assemble HOM definitions in a modular way,
or to construct a more complex bunch pattern.

%\begin{latexonly}
\newpage 
\begin{center}{\tt ringParameters}\end{center}
%%\end{latexonly}
\subsection{ringParameters}

\begin{itemize}
\item type: Setup command.
\item function: Set basic ring parameters necessary to calculate
      synchrotron frequency, bunch length, and so on.  Once the
      namelist is processed and written to the standard output file, a
      list of dependent quantities used in the growth rate
      calculations are written to the standard output file.
\end{itemize}
\begin{verbatim}
&ringParameters
    STRING twissFile = NULL;
    double energyGeV = 0.0
    double circumference = 0.0
    double energyLossPerTurnMeV = 0.0
    double rfVoltageMV = 0.0
    long harmonicNumber = 1
    double momentumCompaction = 0.0
    double relativeEnergySpread = 0.0
    double bunchLengtheningFactor = 1.0
    STRING bunchLengthTableFile = NULL;
    STRING bunchLengthUsedFile = NULL;
    double longDampingTime = 0.0
    double transDampingTime = 0.0
    double horizontalTune = 0
    double verticalTune = 0
    double betaxAtRFCavities = 0
    double betayAtRFCavities = 0
    STRING parameters = NULL;
&end
\end{verbatim}
\begin{itemize}
\item {\tt twissFile} --- If not blank, then parameters of the lattice are
 taken from the named file, which is assumed to be an {\tt elegant} Twiss output
 file. You must turn on the radiation-integral calculation in {\tt elegant} or
 {\tt clinchor} will complain of missing data.  The data taken from the file are
 energy, circumference, energy loss per turn, momentum compaction, relative
 energy spread, damping times, tunes, and beta functions at the cavities.
 For the last item, {\tt clinchor} looks for elements of type {\tt RFCA} and
 {\tt RFCW}.  It averages the beta functions over all elements found.
 If you wish to override the values in the file, simply give a nonzero  
 value to the appropriate namelist variable.
\item {\tt energyGeV} --- Energy of the stored beam in units of GeV.
\item {\tt circumference} --- Circumference of the storage ring in
      meters. Used to calculate revolution frequency $f_0$.
\item {\tt energyLossPerTurnMeV} --- Energy loss per turn in units of MeV.
\item {\tt rfVoltageMV} --- Peak RF voltage in unit of MV of cavities in ring.
\item {\tt harmonicNumber} --- Ratio between external RF frequency
      $f_{\rm RF}$ and the revolution frequency $f_0$.  Used to
      determine $f_{\rm RF}$.
\item {\tt momentumCompaction} --- Momentum compaction factor. Used to
      determine unperturbed bunch length.
\item {\tt relativeEnergySpread} --- Relative energy spread. Used to
      determine unperturbed bunch length.
\item {\tt bunchLengtheningFactor} --- Ratio of expected bunch length
      and unperturbed bunch length, which might be obtained from a
      separate potential well calculation.  This quantity doesn't
      affect the calculation coherent synchrotron frequency. The bunch
      lengthening factor appears only in the bunch form factor in the
      effective HOM impedance expression. This bunch length applies to 
      all bunches in the calculation irrespective of the charge they contain.
\item {\tt bunchLengthTableFile} ---  Input file of bunch length in units of seconds
      as a function of charge to be used for individual bunches in the
      calculation. This bunch length is used as form factor to reduce the
      wake fields for a given resonator. This should be used instead of
      {\tt bunchLengtheningFactor} when possible.
\item {\tt bunchLengthUsedFile} --- Output file of bunch lengths for each
      bunch defined in the calculation. Columns are bucket number, bunch current and 
      bunch length. There will be one row entry for every
      bucket filled. The output files is created only when
      the file {\tt bunchLengthTableFile} is specified. Recommended value is ``{\tt \%s.bl}'' \
      where the ``{\tt \%s}'' is replaced with the root name of the command file.
\item {\tt longDampingTime, transDampingTime} --- Coherent damping
      time constants in seconds for the motions in the longitudinal
      and transverse planes.  If one doesn't know the coherent damping
      times, one may use the synchrotron radiation damping times,
      which are longer. Damping rates are calculated from the damping
      times, and are simply added to the CBM growth rates to give the
      final growth rates.  When doing Monte Carlo studies, I prefer to
      ignore the damping rates by making them explicitly zero in the namelist
      by setting {\tt longDampingTime}, and {\tt transDampingTime} to
      zero.
\item {\tt horizontalTune, verticalTune} --- Tunes of the stored
      beam. Values are required if command {\tt doTransverseMotion} is
      used.
\item {\tt betaxAtRFCavities, betayAtRFCavities} --- Beta functions at
      the RF cavities.  Values are required if command {\tt
      doTransverseMotion} is used.  The transverse growth
      rates will scale with one of the two values, depending on which
      direction is specified in the {\tt doTransverseMotion} command.
\item {\tt parameters} --- Output file containing above parameters
      plus some calculated ones.
\end{itemize}

%\begin{latexonly}
\newpage
\begin{center}{\tt monopoleHOMs}\end{center}
%\end{latexonly}
\subsection{monopoleHOMs}

\begin{itemize}
\item type: Setup command. Required if the next action command is {\tt
      doLongitudinalMotion}.
\item function: Read in a \SDDS\ format file of HOM definitions to
      form or add to an internal monopole HOM data structure array.
      Some of the namelist variables can be used to modify the HOM
      properties once the growth rate calculation starts.
\end{itemize}
\begin{verbatim}
&monopoleHOMs
    STRING filename = NULL,
    long clearPreviousMonopoleHOMs = 0,
    double deQFactorMultiplier = 1 ;
    double Q = 0;
    double staggeringStepMultiplier = 1;
    long detuneFundamental = 0;
    long keepFundamentalFixed = 1;
&end.
\end{verbatim}
\begin{itemize}
\item {\tt filename} --- \SDDS\ format file of HOM definitions to be
      interpreted as monopole HOMs. This is the only required variable
      of the namelist command. The contents of the file is explained
      in section \ref{sect:HOM file definition}.
\item {\tt clearPreviousMonopoleHOMs} --- If this logical flag is set
      to 1, then the current array of monopole HOMs data structure is
      cleared before the file {\tt filename} is read. If the flag is
      0, then the HOM information in the file {\tt filename} is added
      to the current array of monopole HOMs data structure. Thus there
      can be more than one {\tt \&monopoleHOMs} with different data files.
\item {\tt deQFactorMultiplier} --- Factor by which all $Q$'s in the 
      file will be divided before running a calculation.
\item {\tt Q} --- All HOMs of this file will be assigned the value of $Q$
      specified here.
\item {\tt staggeringStepMultiplier} --- If there is a frequency staggering step
      quantity in the HOM file, then this factor is used to increase the 
      staggering step before a calculation is done.
\item {\tt detuneFundamental} --- If nonzero, then {\tt clinchor} looks in
 the HOM list for the mode that most closely matches the fundamental rf
 frequency (computed from the circumference and the harmonic number).
 The program sets that HOM frequency to the fundamental rf frequency, then
detunes it to compensate beam loading.  
\item {\tt keepFundamentalFixed} ---  If nonzero, then the fundamental frequency will not be randomized along with other HOMs. 
This is to simulate what is done in practise, that the fundamental mode
tuning is kept fixed by feedback loops.
This fixing of fundamental mode can be done through the data column ``Fixed'' in the HOM file as well.
\end{itemize}

%\begin{latexonly}
\newpage
\begin{center}{\tt dipoleHOMs}\end{center}
%\end{latexonly}
\subsection{dipoleHOMs}

\begin{itemize}
\item type: Setup command. Required if the next action command is {\tt
      doTransverseMotion}.
\item function: Read in a \SDDS\ format file of HOM definitions to
      form or add to an internal dipole HOM data structure array.
      Some of the namelist variables can be used to modify the HOM
      properties once the growth rate calculation starts.
\end{itemize}
\begin{verbatim}
&dipoleHOMs
    STRING filename = NULL
    long clearPreviousDipoleHOMs = 0
    double deQFactorMultiplier = 1;
    double Q = 0;
    double staggeringStepMultiplier = 1;
    long detuneFundamental = 0;
&end.
\end{verbatim}
\begin{itemize}
\item {\tt filename} --- \SDDS\ format file of HOM definitions to be
      interpreted as dipole HOMs. This is the only required variable
      of the namelist command. The contents of the file is explained
      in section \ref{sect:HOM file definition}.
\item {\tt clearPreviousDipoleHOMs} --- If this logical flag is set to
      1, then the current array of dipole HOMs data structure is
      cleared before the file {\tt filename} is read. If the flag is
      0, then the HOM information in the file {\tt filename} is added
      to the current array of dipole HOMs data structure. Thus there
      can be more than one {\tt \&monopoleHOMs} with different data files.
\item {\tt deQFactorMultiplier} --- Factor by which all $Q$'s in the 
      file will be divided before running a calculation.
\item {\tt Q} --- All HOMs of this file will be assigned the value of $Q$
      specified here.
\item {\tt staggeringStepMultiplier} --- If there is a frequency staggering step
      quantity in the HOM file, then this factor is used to increase the 
      staggering step before a calculation is done.
\item {\tt detuneFundamental} --- If nonzero, then {\tt clinchor} looks in
      the dipole HOM list for the mode with the lowest frequency. 
      The program sets that HOM frequency to the closest harmonic of the
      fundamental rf frequency, then detunes it to compensate beam loading.  
      This has applications only for a deflection-mode cavity running
      at the harmonic of the rf frequency.
\end{itemize}

%\begin{latexonly}
\newpage
\begin{center}{\tt symmetricBunchPattern}\end{center}
%\end{latexonly}
\subsection{symmetricBunchPattern}

\begin{itemize}
\item type: Setup command. 
\item function: This command defines a symmetric bunch pattern.
\end{itemize}
\begin{verbatim}
&symmetricBunchPattern
    long startBucket = 0
    long bunches = 1
    double currentPerBucketMA = 0.0
    double totalCurrentMA = 0.0
    long clearPreviousPatterns = 0
&end.
\end{verbatim}
\begin{itemize}
\item {\tt startBucket} --- Bucket number of the first bunch in the
bunch train. In the case of a symmetric bunch pattern, {\tt
startBucket} is the bucket number where one of the bunches is
located. Default is 0, the first bucket.
\item {\tt bunches} --- Number of bunches in the symmetric
distribution. If {\tt bunches} is not a divisor of the RF harmonic
number, then a close approximation of the symmetric distribution is
generated.
\item {\tt currentPerBucketMA} --- Current in mA for each bunch
defined.
\item {\tt totalCurrentMA} --- Total current of all bunches defined
in this command. Used to determine the current of individual
bunches. Either variables {\tt currentPerBucketMA} or {\tt
totalCurrentMA} may be defined.  If both are present, variable {\tt
currentPerBucketMA} takes precedence.
\item {\tt clearPreviousPatterns} --- If this flag is set to 1, then
bunch patterns defined in previous {\tt symmetricBunchPattern} or
{\tt generalBunchPattern} are cleared before setting the bunches of
this namelist command.
\end{itemize}

%\begin{latexonly}
\newpage
\begin{center}{\tt bunchTrain}\end{center}
%\end{latexonly}
\subsection{bunchTrain}

\begin{itemize}
\item type: Setup command. 
\item function: This command defines a train of equally spaced bunches of equal charge.
\end{itemize}
\begin{verbatim}
&bunchTrain
    long startBucket = 0
    long bucketInterval = 1
    long bunches = 1
    double currentPerBucketMA = 0.0
    double totalCurrentMA = 0.0
    long clearPreviousPatterns = 0
&end.
\end{verbatim}
\begin{itemize}
\item {\tt startBucket} --- Bucket number of the first bunch in the bunch train.
\item {\tt bucketInterval} --- Bunch spacing interval for a train of equally spaced train of bunches.
The number of empty buckets in between the bunches is {\tt bucketInterval}-1.
\item {\tt bunches} --- Number of bunches in the train.
\item {\tt currentPerBucketMA} --- Current in mA for each bunch defined.
\item {\tt totalCurrentMA} --- Total current of all bunches defined
in this command. Used to determine the current of individual
bunches. Either variables {\tt currentPerBucketMA} or {\tt
totalCurrentMA} may be defined.  If both are present, variable {\tt
currentPerBucketMA} takes precedence.
\item {\tt clearPreviousPatterns} --- If this flag is set to 1, then
bunch patterns defined in all previous bunch defining commands are
cleared before creating the bunches of this namelist command.
\end{itemize}

%\begin{latexonly}
\newpage
\begin{center}\tt generalBunchPattern\end{center}
%\end{latexonly}
\subsection{generalBunchPattern}

\begin{itemize}
\item type: Setup command.
\item function: The command defines a general distribution of bunches
    using STRING variables containing a list of bunch positions and
    current values.
\end{itemize}
\begin{verbatim}
&generalBunchPattern
    STRING bucketSelection = NULL
    STRING currentPerBucketMA = NULL
    double totalCurrentMA = 0.0
    long clearPreviousPatterns = 0
&end.
\end{verbatim}
\begin{itemize}
\item {\tt bucketSelection} --- String containing a list of bucket
      numbers (long integers) for bunches to fill.  A string variable
      is necessary for this input because a string has no pre-set
      length, while an array variable must be defined with a pre-set
      length.
\item {\tt currentPerBucketMA} --- String containing a list of bunch
      currents (double floating point numbers) corresponding to each
      bucket number specified in {\tt bucketSelection}.  If the
      variables {\tt bucketSelection} and {\tt currentPerBucketMA}
      have unequal number of entries, then the longer one is
      truncated.  Here is an example of the use of the string
      variables for generating some arbitrary general pattern:
\begin{verbatim}
&generalBunchPattern
    bucketSelection = "1 4 9 16"
    currentPerBucketMA = "1.0 2.0 3.0 4.0"
&end.
\end{verbatim}
\item {\tt totalCurrentMA} --- Total current to be distributed
evenly among bunches defined in {\tt bucketSelection}.  This variable
is used when variable {\tt currentPerBucketMA} is not specified.
\end{itemize}

%\begin{latexonly}
\newpage
\begin{center}\tt bunchPatternFromFile\end{center}
%\end{latexonly}
\subsection{bunchPatternFromFile}

\begin{itemize}
\item type: Setup command.
\item function: The command reads in a file to define a presumably
    general distribution of bunches. This can be the {\tt
    bunchLenghtUsedFile} output file produced by a previous run of
    {\tt clinchor}, since that file contains the bucket number and
    bunch current.
\end{itemize}
\begin{verbatim}
&generalBunchPattern
    STRING filename = NULL
    long clearPreviousPatterns = 0
&end.
\end{verbatim}
\begin{itemize}
\item {\tt filename} --- Input file with column Bucket and Current (mA).
\end{itemize}


%\begin{latexonly}
\newpage
\begin{center}\tt randomizeBunchCurrent\end{center}
%\end{latexonly}
\subsection{randomizeBunchCurrent}

\begin{itemize}
\item type: Setup command.
\item function: The command defines the randomization parameters for
the bunch pattern. The randomization is performed on the ideal bunch
pattern before each growth rate calculation. If a sweep of frequency
is requested ({\tt sweepFrequency}), then a new random bunch pattern
is generated for each HOM frequency value. If a HOM-frequency Monte
Carlo simulation is requested ({\tt randomizeHOMFrequencies}), then a
new random bunch pattern is generated for each HOM frequency
sample. Thus there are two random components in the resulting growth
rate distribution: HOM frequency and bunch population.
\end{itemize}
\begin{verbatim}
&randomizeBunchCurrent struct
    long seed = -987654323;
    long uniform = 1;
    double absoluteSpreadMA = 0;
    double relativeSpread = 0;
&end.
\end{verbatim}
\begin{itemize}
\item {\tt seed} --- A large negative number for the random number generator. This will use an independent generator from the HOM frequency.
\item {\tt uniform} --- Flag for uniform distribution. Presently only uniform distribution is available.
\item {\tt absoluteSpreadMA} --- Spread in bunch distribution in units of mA. This is useful in top-up mode injection where the total current is constant but
individual bunches have spread equal to the charge per injection bunch. For example if a bunch pattern has 4 mA per bunch and the spread is 1 mA, then some  bunches may have current as low as 3.5 mA and others as high as 4.5 mA. 
\item {\tt relativeSpread} --- Spread in bunch distribution in fraction.
\end{itemize}


%\begin{latexonly}
\newpage
\begin{center}\tt sweepFrequency\end{center}
%\end{latexonly}
\subsection{sweepFrequency}

\begin{itemize}
\item type: Setup command
\item function: Sets up a sweep of the resonator frequency value of a
selected HOM. The setup takes effect only when an action command is
executed.
\end{itemize}
\begin{verbatim}
&sweepHOMFrequency
    long resonatorIndex = 0
    double frequencyRange = 0.0
    long points = 100
    STRING filename = NULL
&end
\end{verbatim}
\begin{itemize}
\item {\tt resonatorIndex} --- HOM resonator number to be swept in
frequency. The numbering starts at 0, and proceed in the order that
the resonators were read in the HOM definition file or files.
\item {\tt frequencyRange} --- Sweep range in Hz. The range starts
at the unperturbed HOM resonator frequency minus {\tt
frequencyRangeHz} and ends at the unperturbed HOM resonator frequency
plus {\tt frequencyRangeHz}. If {\tt frequencyRangeHz} is set to 0 or
not specified, then {\tt frequencyRangeHz}=$f_0$ where $f_0$ is the
revolution frequency.
\item {\tt points} --- Number of points in the sweep.
\item {\tt filename} --- \SDDS\ file in which the results are
written. The columns are the swept HOM frequency, the maximum growth
rate, and the frequency shift of the fastest growing CBM. Recommended
naming conventions: {\tt <something>.CBMFreq}
\end{itemize}

%\begin{latexonly}
\newpage
\begin{center}\tt randomizeHOMFrequencies\end{center}
%\end{latexonly}
\subsection{randomizeHOMFrequencies}

\begin{itemize}
\item type: Setup command
\item function: Sets up a Monte Carlo simulation of randomized HOM
frequencies for all HOMs. The setup takes effect only when an action
command is executed.
\end{itemize}
\begin{verbatim}
&randomHOMFrequencies
    double spread = 0.0
    long seed = -987654321
    long uniform = 1
    long samples = 100
    STRING CBMFrequencyFilename = NULL
    STRING HOMFrequencyFilename = NULL
&end
\end{verbatim}
\begin{itemize}
\item {\tt spread} --- The range of the random component of the
frequency values is -{\tt spread} to {\tt spread} in Hz.  If {\tt spread} is set
to zero or left undefined, then {\tt spread} = $f_0$ where $f_0$ is the
revolution frequency.
\item {\tt seed} --- A large negative number for the random number generator. This will use an independent generator from the bunch population randomizer.
\item {\tt uniform} --- If flag is set to 1, then a uniform
distribution for the random component of the HOM frequencies is used.
Since no other random distribution type is available in this present
version of \clinchor\, using other values of {\tt uniform} makes no
difference.
\item {\tt samples} --- Gives the number of times a following action
command ( {\tt doLongitudinalMotion} or {\tt doTransverseMotion})is
executed with a different sampling of random HOM frequencies for each
calculation.
\item {\tt CBMFrequencyFilename} --- File containing the growth rate value of the
fastest growing mode in each Monte Carlo sample.  This file can then
be histogram med with the \SDDS\ toolkit program \sddshist. Recommended
naming conventions: {\tt <something>.CBMFreq}

\item {\tt HOMFrequencyFilename} --- \SDDS\ file containing the
randomized HOM frequencies used by each Monte Carlo sample. Type ``{\tt
sddsquery <filename>}'' for the full description of the file.  This
file may grow to be very large, so be careful with this option. This
variable is optional. Recommended naming conventions: {\tt <something>.HOMFreq}
\end{itemize}

%\begin{latexonly}
\newpage
\begin{center}\tt doLongitudinalMotion\end{center}
%\end{latexonly}
\subsection{doLongitudinalMotion}

\begin{itemize}
\item type: Action command
\item function: Does calculation of longitudinal CBM growth rates
using monopole HOMs and bunch definitions currently defined.  Monopole
HOMs and a bunch pattern must have been defined previously for this
command to work.  Either one of the optional commands {\tt
sweepFrequency} and {\tt randomizeHOMFrequencies} may appear before.  If both
appear, the last one takes precedence.

If the optional command {\tt sweepFrequency} is in effect, the
\SDDS\ file specified by the {\tt filename} variable in that namelist
command is opened at this point.

If the optional command {\tt randomizeHOMFrequencies} is in effect,
the \SDDS\ files specified by the {\tt filename} and {\tt
frequencyListFilename} variables in that namelist command are opened
at this point.
\end{itemize}
\begin{verbatim}
&doLongitudinalMotion
   long normalModes = 1
   long doLaplace = 0
   STRING eigenvectorFilename = NULL;
   STRING CBMFrequencyFilename = NULL
&end
\end{verbatim}
\begin{itemize}
\item {\tt normalModes} --- If set to 1, then CBM complex frequencies
        are calculated, which is the only present goal of the program.
\item {\tt doLaplace} --- Not yet available. This solves the motion
as a function of time with initial conditions using an inverse Laplace
transform, as explained in \cite{Thompson:PAC89}.
\item {\tt eigenvectorFilename} --- Print the normal
mode matrix used in the CBM frequencies calculation. Recommended
naming conventions: {\tt <something>.v}
\item {\tt CBMFrequencyFilename} --- \SDDS\ file listing the
complex frequencies of all the CBMs. This file may grow very large
with the Monte Carlo simulation.   Recommended
naming conventions: {\tt <something>.CBMFreq}

\end{itemize}

%\begin{latexonly}
\newpage
\begin{center}{\Large\tt doTransverseMotion}\end{center}
%\end{latexonly}
\subsection{doTransverseMotion}

\begin{itemize}
\item type: Action command
\item function: Does calculation of transverse CBM growth rates using
monopole HOMs and bunch definitions currently defined.  Of course,
monopole HOMs and a bunch pattern must have been defined previously
for this command to work.  Either one of the optional commands {\tt
sweepFrequency} and {\tt randomizeHOMFrequencies} may appear.  If
both appear, the last one takes precedence.

If the optional command {\tt sweepFrequency} is in effect, the
\SDDS\ file specified by the {\tt filename variable} in that namelist
command is opened at this point.

If the optional command {\tt randomizeHOMFrequencies} is in effect,
the \SDDS\ files specified by the {\tt filename} {\tt
frequencyListFilename} variables in that namelist command are opened
at this point.
\end{itemize}
\begin{verbatim}
&doTransverseMotion
   long normalModes = 1
   long doLaplace = 0
   long verticalDirection = 0
   long horizontalDirection = 0
   STRING CBMFrequencyFilename = NULL;
   STRING eigenvectorFilename  = NULL;
&end
\end{verbatim}
\begin{itemize}
\item {\tt normalModes} --- If set to 1, then CBM complex frequencies
are calculated, which is the present goal of the program.
\item {\tt doLaplace} --- Not yet available. This solves the motion
as a function of time with initial conditions using an inverse Laplace
transform, as explained in \cite{Thompson:PAC89}.
\item {\tt verticalDirection} --- If flag is set to 1, then the
vertical tune and the vertical beta function at the RF cavities (see
variables of {\tt ringParameters}) are used in the calculation.
\item {\tt horizontalDirection} --- If flag is set to 1, then the
horizontal tune and the horizontal beta function at the RF cavities
(see variables of {\tt ringParameters}) are used in the calculation.
\item {\tt eigenvectorFilename} --- Print the normal
mode matrix used in the CBM frequencies calculation.
\item {\tt CBMFrequencyFilename} --- \SDDS\ file listing the
complex frequencies of all the CBMs. Recommended
naming conventions: {\tt <something>.CBMFreq}
\end{itemize}

%\begin{latexonly}
\newpage
\begin{center}{\tt stop}\end{center}
%\end{latexonly}
\subsection{semaphores}

\begin{itemize}
\item type: Action command
\item function: Creates small files upon start, failure or normal exit, the presence of which can be used as signals for scripts to monitor a batch of simulations.
\end{itemize}
\begin{verbatim}
&semaphores
STRING started = "%s.started";
STRING done = "%s.done";
STRING failed = "%s.failed";
&end
\end{verbatim}



%\begin{latexonly}
\newpage
\begin{center}{\tt stop}\end{center}
%\end{latexonly}
\subsection{stop}

\begin{itemize}
\item type: Action command
\item function: Stops the execution of the program. No variable are defined.
\end{itemize}
\begin{verbatim}
&stop
&end
\end{verbatim}

\section{HOM Definition SDDS File Format} \label{sect:HOM file definition}
The general \SDDS\ format is defined by two parts, the header and zero
or more data tables.  The header contains namelist commands defining,
among other things, the interpretation of the data in the tables. A
table consists of a list of parameters values followed by columns of
data. In the HOM definition file, \clinchor\ requires the presence of
columns {\tt Frequency}, and any two of {\tt ShuntImpedance}, {\tt Q},
{\tt RoQ} (the ratio $R/Q$). 
For many columns of physical quantities the correct units must be specified.

The allowed column names for the HOM definition file are: 
\begin{itemize}
\item{\tt Frequency} or {\tt f} --- Required column. Frequency in
    units of ``Hz'' or ``MHz''. The program internally converts to Hz.
\item{\tt ShuntImpedance} or {\tt R} or {\tt Rs} or {\tt Rt} ---
    Required column (see above).  Shunt impedance in units of ``Ohm'' for
    monopole HOMs, and units of ``Ohm/m'' for dipole HOMs.
\item{\tt Q} or {\tt QualityFactor} --- Required column (see above). Quality factor of the resonator.
\item{\tt ROverQ} or {\tt R/Q} or {\tt ROQ} --- Required column (see
    above). Shunt impedance divided by $Q$. In units of ``Ohm'' for
    monopole HOMs, and units of ``Ohm/m'' for dipole HOMs.
\item{\tt DeQFactor} or {\tt deQFactor} --- Factor by which to reduce the
  $Q$ of the HOM after reading in this file.
\item{\tt StaggeringStep} or {\tt staggeringStep} or {\tt
DeltaFrequency} or {\tt deltaFrequency} or {\tt df} --- Optional
column. Units of Hz only. Default value is 0. If larger than one, then the
frequencies of the {\tt NumberOfCavities} HOM resonators are staggered
by this interval.
\item{\tt ShiftToResonance} --- Optional column. Default value is
0. If non-zero, then the frequency of the HOMs is shifted to the
closest resonance. If {\tt staggeringStep} is non-zero, then the
frequency shifting is done after staggering the frequencies.

\item{\tt NumberOfCavities} or {\tt NCavities} or {\tt Cavities} or
  {\tt cavities} --- Optional parameter. Default value is 1. If 0,
  then this HOM type is not used in the calculation.   If greater than one,
then the HOM resonator is duplicated internally {\tt numberOfCavities}
times. 
All resonators will be subjected to randomized frequencies separately (if
requested by the {\tt randomizeHOMFrequencies} command.) 
\end{itemize}


Here is an example of an \SDDS\ HOM definition file for the APS ring cavities:
\begin{verbatim}
SDDS1
&description text="APS cavity longitudinal HOMs" contents="HOM definition" &end
&column name=Frequency, symbol=f, units=Hz, type=double, 
        description="Resonant frequency of HOM resonator" &end
&column name=ShuntImpedance, symbol=R, units=Ohm, type=double,
        description="Shunt impedance of longitudinal HOM resonator" &end
&column name=Q, symbol=Q, units="", type=double,
        description="Q of longitudinal HOM resonator" &end
&column name=NumberOfCavities, symbol=N$bcav$n, units="", type=long,
        description="Number of cavities for each HOM" &end
&column name=StaggeringStep, units=Hz, type=double,
        description="Staggering frequency steps between cavity HOMs" &end
&column name=ShiftToResonance, type=long,
        description="flag causes automatic shift to resonance after Staggering" &end
&data mode=ascii, noRowCounts=1 &end
558.7e6   13.6e6   68e3    16        -0.08e6   0
761.1e6   25.6e6   53e3    16        -0.7e6    0
962.0e6    6.1e6   54e3    16        -1.2e6    0
1017.4e6   2.6e6   41e3    16        -1.7e6    0
1145.1e6   2.7e6   92e3    16        -1.5e6    0
1219.2e6   3.6e6   41e3    16        -1.9e6    0
\end{verbatim}

\section{Example Files}
Example \clinchor\ input files, runs of \clinchor\ and post-processing commands are 
included with the distribution of \clinchor.
The examples can be run with the script files provided. 

\section{Acknowledgment}
The structure of the program is modeled after the program \elegant\ written by M. Borland. The program
also links to libraries written by M. Borland.

\bibliographystyle{ieeetr}
\bibliography{coupledBunch}
\begin{figure}[p]
     \caption{Example of histogram} \label{fig:histogram-example}
\end{figure}
\begin{figure}
    \caption{Examples of postprocessing of the frequency list file} \label{fig:postprocess-examples}
\end{figure}
\begin{figure}
    \caption{Example of HOM frequency sweep} \label{fig:sweep-example} 
\end{figure}
\end{document}
